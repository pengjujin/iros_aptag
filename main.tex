\newcommand{\CLASSINPUTtoptextmargin}{0.75in}
\newcommand{\CLASSINPUTbottomtextmargin}{0.75in}
\newcommand{\CLASSINPUTinnersidemargin}{0.75in}
\newcommand{\CLASSINPUToutersidemargin}{0.75in}
\documentclass[letterpaper,conference,10pt]{ieeetran}
\IEEEoverridecommandlockouts
\overrideIEEEmargins
\pdfoptionpdfminorversion=4
%\usepackage[sc]{mathpazo}
%\linespread{1.02}         % Palatino needs more leading (space between lines)
\usepackage[T1]{fontenc}


\usepackage{multicol}
% Insert hyperlinks in the document
\usepackage{hyperref}
\hypersetup{bookmarksopen,bookmarksnumbered,
pdfpagemode=UseOutlines,
colorlinks=true,
linkcolor=blue,
anchorcolor=blue,
citecolor=blue,
filecolor=blue,
menucolor=blue,
urlcolor=blue
}
\usepackage{amsmath, amssymb, amsfonts, amsthm}
\usepackage{bm}
%\usepackage{subcaption}
\usepackage{flushend}
\usepackage{dirtree}
\usepackage{graphicx}
\usepackage{mathtools}
\usepackage[numbers,sort&compress]{natbib}
\usepackage{fixltx2e}

% Labels in IEEE format
\newcommand{\eref}[1]{(\ref{#1})} % Equation
\newcommand{\sref}[1]{Sec. \ref{#1}} % Section
\newcommand{\figref}[1]{Fig.\ref{#1}} % Figure
\newcommand{\tref}[1]{Table \ref{#1}} % Table



% floats
\usepackage[font=footnotesize]{caption} % incompatible with aaai.sty?
%\usepackage[font=footnotesize]{subcaption}
\usepackage{mathabx}
\usepackage{verbatim}
\usepackage{algorithm}
\usepackage{amsmath}
\usepackage[noend]{algpseudocode}


\usepackage{multirow}
\usepackage{booktabs}
\usepackage[dvipsnames]{xcolor}

\setlength{\textfloatsep}{5pt}
\setlength{\intextsep}{5pt}

\newcommand{\ssnote}[1]{{\xxnote{SS}{red}{#1}}}
\newcommand{\scnote}[1]{{\xxnote{SC}{blue}{#1}}}
\newcommand{\cdnote}[1]{{\xxnote{CD}{cyan}{#1}}}
\newcommand{\xxnote}[3]{}
\ifx\hidenotes\undefined
 \usepackage{color}
 \renewcommand{\xxnote}[3]{\color{#2}{#1: #3}}
\fi

\ifCLASSOPTIONcompsoc
  \usepackage[caption=false,font=normalsize,labelfont=sf,textfont=sf]{subfig}
\else
  \usepackage[caption=false,font=footnotesize]{subfig}
\fi

\DeclareMathOperator*{\argmin}{\arg\!\min}

\begin{document}

\title{\vspace{0.20in}\LARGE A Sensor Fusion Approach For }

\author{Pengju Jin, Pyry Matikainen, Siddhartha Srinivasa}
\maketitle 	

\begin{abstract}
Although there is an abundance of planar fiducial-marker systems proposed for augmented reality and computer-vision purposes, none of them are robust enough to estimate the pose accurately in robotic applications where tags are small and the collected data are noisy. This is inherently a difficult problem because these fiducial marker systems work solely within the RGB image space and the resolution of cameras on robots is constrained. As a result, small noise in the image would cause the tag's detection process to produce large pose estimation errors. 

This paper describes an algorithm that improves the pose estimation accuracy of square fiducial markers in difficult scenes by fusing information from RGB and depth sensors. The algorithm retains the high detection rate and low false  positive rate characteristics of fiducial systems while making them much more robust to size, lighting and sensory noise for pose estimation. The improvements make the fiducial tags suitable for robotic tasks requiring high pose accuracy in the real world environment.
\end{abstract}

\IEEEpeerreviewmaketitle

% !TEX root = ../main.tex
\section{Introduction}
\label{sec:intro}
Detection and identification using artificial landmarks has long been popular in augmented reality and computer vision. Planar markers such as ARTag, ARToolkil, are among the most popular forms of fiducial markers. Compared to markerless detection algorithms, fiducial tags are much more reliable. High detection rates and accurate encodings make them popular in robotic applications such as testing SLAM systems, finding ground truth for manipulation tasks.    

In robotic applications, it is important to be able to accurately and reliably calculate the pose of the fiducial tags. There has been large amounts of effort for improving the detection algorithm by making them faster and more accurate using the RGB images. These algorithm yield great results under well conditioned or simulated environments. (Provide reference to specific studies). However, in real robotic application, these algorithms are often tested under various lighting and sensory noises. In these conditions, the fiducial tags suffers greatly from the perceptual ambiguity problem and makes the pose estimation difficult without additional information. In fact, we observe that the localization accuracy of the state of the art Apriltags is significantly worse at difficult viewing angles or when there are noise in the scene. 

In this paper, we present an algorithm that take advantage of the RGBD sensor to accurately estimate the pose from a single tag under noisy conditions. There are few key features to this algorithm: 
\begin{itemize}
\item The algorithm is generalizable to all square based fiducial tags.
\item The algorithm performs at worse as good as only using RGB images.
\end{itemize}

% !TEX root = ../main.tex
\section{Related Work}
\label{sec:related}
\begin{figure}
\centering
\subfloat[ARToolkit]{\includegraphics[width=50px, height=50px]{figs/tags/artoolkit}} \quad
\subfloat[ARTag]{\includegraphics[width=50px, height=50px]{figs/tags/artags}} \quad 
\subfloat[AprilTag]{\includegraphics[width=50px, height=50px]{figs/tags/apriltags}} \\
\subfloat[RUNE-Tag]{\includegraphics[width=50px, height=50px]{figs/tags/rune}} \quad
\subfloat[Intersense]{\includegraphics[width=50px, height=50px]{figs/tags/intersense}}
\caption{Different types of popular fiducial tags. ARToolkit, ARTags, and AprilTags are square tags with black borders. RUNE-tags and Intersense use different circle features as landmarks}
\label{fig:tags}
\end{figure}
	Obtaining highly accurate pose estimation has been an important research area in robotics. Numerous algorithms rely only on RGB or gray scale images. Solving the projection geometry of some detected features and then minimize the reprojection error of the features in the image space \citep{grest2009comparison}.  Similarly, methods such as Iterative Closet Point \citep{besl1992method} were developed to solve the pose estimation problem using range data by minimizing the Euclidean distance between the model and the depth data. Recently, some approaches propose to enhance the accuracy of traditional algorithms by fusing RGB and depth data in various problems using extended Kalman filters \citep{gedik2015rgbd, assa2014robust}. Compared to the single-sensor approaches, algorithms utilizing RGBD data are more accurate and perform well in noisy situations where other approaches fail. 
	
	Fiducial markers solve pose estimation by exploiting easily detectable features in the RGB space. Although there is an abundance of unique tag designs, most of them carry easily recognizable yet precise binary patterns in the inner region to encode information. There are two types of common tags: circular tags and square tags seen in Figure \ref{fig:tags}. 
	
	Circular tags are created to encode the payload using small circular patterns arranged in various shapes. The example of circular tags include Intersense \citep{naimark2002circular} and Rune tags \citep{bergamasco2011rune}. The perspective transformation of a circle is an ellipse, which can be used to directly compute the pose using back projection methods. Localization of circular features is generally more accurate, and thus generates better pose estimation at the cost of higher computation time \citep{rice2006analysing}. However, small circular features become hard to detect when they are far away from the camera or prospectively rotated, and thus their effective range is much smaller than that of square tags. This characteristic makes them less useful in applications with size constraints. 
	
	ARTags \citep{fiala2004artag}, ARToolkit \citep{kato2002artoolkit}, AprilTag \citep{olson2011apriltag} and AprilTag 2 \citep{wang2016apriltag} are examples of squared-based fiducial tags. The perspective projection of a square becomes a general quadrilateral, which can be computed easily using any contour tracing algorithm. Given the scale of a single marker, the full 6-DOF pose can then be estimated using the corners of the quadrilateral. However, since the tags are detected using rectangles and lines, the accuracy of their corner point sub-pixel locations is limited. Among the square tags, ARToolkit is one of the earliest detection systems, and it is mainly used for Augmented reality applications. Instead of using a binary payload, it used various symbols to encode the tag, and it was computationally expensive to decode the tag. Built on top of ARToolkit, ARTags and Apriltag reduced the computation time by using a 2D binary pattern as the payload. Both systems use the image gradient to compute the tag border making it robust to lighting changes and partial occlusions. Relative to ARTags, Apriltags have a lower false positive rate, as they use a lexicode-based system that is invariant to rotation. In addition, Apriltags have higher detection rates at further distances and at more difficult viewing angles. Recently AprilTag 2 improved upon the original Apriltag. It implements a new boundary segmentation algorithm which further reduces the computing time for detection and increases the detection rate. Compared to circular tags, the advantages of square tags are that they can be located very efficiently and they have reliable decoding schemes. Therefore, even though the square tags have slightly lower localization accuracy, they are more suitable for robotic applications that require a robust system.

% !TEX root = ../main.tex
\section{Perceptual Ambiguity}
\label{sec:problem}
In most square fiducial tag detection, the pose of the tag is calculated from the quad fitted around the tag. The corners are extracted from the tag and the pose of the tag is estimated using some 3D to 2D point correspondence optimization. However, small variance in the corner detection process will yield estimations far from the true pose due to a perceptual ambiguity under perspective projection. 

We will illustrate this effect by using two over lapping cubes in figure 3. The marked face of both cubes are interlaced and oriented off by 120 degrees. However, due to perspective projection the squares appears to be on the same plane. Note that in theory, the perfect (Back this up with formula) perspective projection of a square should be unique. However, the projected square of the above set up are close to each other. Therefore, under noise, the two projected squares become indistinguishable. The result of the 3D to 2D correspondence optimization might return either one of the two solution.
\begin{figure}
\centering
\includegraphics[width=\columnwidth]{figs/mismatch_tag}
\caption{The orientation of Apriltag placed on the object is greatly misaligned with the actual object}
\label{fig:calib}
\end{figure}

\begin{figure}
\centering
\includegraphics[width=\columnwidth]{figs/perspective_fig}
\caption{Perspective Ambiguity illustrated with overlapping cubes}
\label{fig:calib}
\end{figure}

% !TEX root = ../main.tex
\section{Approach}
\label{sec:approach}
This section describes a method for accurately estimating poses for square fiducial tags in noisy settings by fusing RGBD data. The process of detecting and decoding the tag is identical to previous fiducial tag systems. After the tag corners are detected, they are treated as approximated locations of the true corners. Using the corners, the method implicitly evaluate the depth data and RGB data as two separate observations and fuse them to minimize the error in 2D and 3D space.

There are three distinct components to this method. First, we find the plane in $SO(3)$ containing the tag using depth data and detected corners. Secondly, an approximate initial pose is computed using the depth plane. Finally, the method refines the initial pose using the RGB data by minimizing the reprojection error within a constrained space. Each component is described in detail in the following subsections. 

\subsection{Depth Plane Fitting}
The first step is to extract the plane which the tag is laying on. We assume that the RGBD sensor is calibrated such that depth and RGB streams are registered to the same frame. The rectangular patch of points in the depth image bounded by the approximated corner pixels $\boldsymbol{y} = [y_1, y_2, y_3, y_4]$ contains the range information of all the points on the tag. Here we take advantage of the planar characteristic of the tag. By fitting a plane over the range data, we can constrain the pose of the tag to be on the plane.

The raw range data retrieved from the depth sensors are generally noisy. The borders and dark regions of the tag produce unreliable range data and artifacts due to a weakness of our depth sensors (time of flight sensor from Kinect V2). Therefore, we first filter the data by removing points too far from the median before fitting the plane. Nevertheless, the remaining points could have a large variance depending on the lighting condition and the magnitude of the in-plane rotation. The accuracy of the plane fit and initial pose estimation is directly affected by the noise level of data. We will characterize the uncertainty of the plane fit and adjust the weight of the initial estimation accordingly during the fusing stage.

In implementation, we used a Bayesian plane fitting algorithm described in \citep{pathak2010uncertainty} which computes the Hessian Normal parameters $[\boldsymbol{\hat{n}}, d]$ of a plane for noisy range data through optimizing
\begin{IEEEeqnarray}{c}
\min _{\boldsymbol{\hat{n}}, d} \sum_{j=1}^{N} 
	\frac{(p_j(\boldsymbol{\hat{n}} \cdot \boldsymbol{\hat{m}_j}) -d)^2}
		 {(\boldsymbol{\hat{n}} \cdot \boldsymbol{\hat{m}_j})^2\sigma ^2\{\bar{p}_j \} }
\label{eq:gaussian_noise}
\end{IEEEeqnarray}
where $\boldsymbol{\hat{n}}$ is the local normal to the planar surface of the depth point and $\boldsymbol{\hat{m_j}}$ is the measurement direction for the sensor for point $p_j$. 
The algorithm in the paper assumes a radial Gaussian noise in the range data $p_j$ with the standard deviation modeled by a function in the form
\begin{IEEEeqnarray}{c}
\sigma \{ \bar{p_j} \} = \frac{kd^2}{ \| \boldsymbol{\hat{n}} \cdot \boldsymbol{\hat{m}_j} \| } 
\IEEEeqnarraynumspace
\label{eq:gaussian_noise}
\end{IEEEeqnarray}
The coefficient $k > 0$ is an estimated value obtained from sensor calibration. In our implementation, we obtained $k$ by using the Kinect V2 model obtained from [Kinect Noise Model paper]. 

An important result we used from \citep{pathak2010uncertainty} is the covariance matrix for the plane-parameters. The covariance is obtained by taking the \textit{Moore-Penrose generalized inverse} of Hessian matrix computed from the Lagrangian. It characterizes the uncertainty of the plane fit and implicitly measures the relative accuracy of the depth data.

\begin{figure}
\centering
\includegraphics[width=\columnwidth]{figs/optimization_visualization_fig}
\caption{An abstract visualization of the optimization constraints. The blue curve is the initial pose estimation obtained from the depth plane. The red curves are the ambiguous poses from the RGB image. We constrained the region of optimization based on how well we fit the depth plane.}
\label{fig:optimization}
\end{figure}

\subsection{Initial Pose Estimation}
The 6 DOF pose of the tag can be described as the transformation $[R, \boldsymbol{t}]$ aligning the tag frame's coordinate system and the sensory frame's coordinate system of the robot. The depth plane $D [ \boldsymbol{\hat{n}}, d]$ alone is insufficient to determine the transformation as it only defines 3 DOF. Since the depth plane was computed base on the approximate center of the tag, we can use the center of the tag and center of the plane as a pair point correspondence. However, there are still infinite number of valid poses rotating about the normal $\boldsymbol{\hat{n}}$. One solution is to constrain the pose by using a corner as an extra point correspondence to solve for the optimal rotation. In practice, the accuracy of this method largely depends on the depth accuracy of the chosen corner point. 


\begin{figure}
\subfloat[RGB]{\includegraphics[width=125px, height=88px]{figs/rgb_result}}
\subfloat[RGBD]{\includegraphics[width=125px, height=88px]{figs/rgbd_result}}
\caption{The pose of the Apriltag visualized in RViz computed using the original library VS our RGBD fused method.}
\label{fig:result_compare}
\end{figure}

\begin{figure*}[h]
\subfloat[RGB image at 60$^{\circ}$]{\includegraphics[width=\columnwidth, height=160px]{figs/result_figs/rgb_smallfig}
\label{fig:exp_setup}}
\subfloat[Rotation errors across 1000 trials]{\includegraphics[width=\columnwidth, height=170px]{figs/result_figs/result_0005}
\label{fig:bimodal}} \\
\caption{An example of the experimental setup in \ref{fig:exp_setup}. Groundtruth is computed from a large chessboard where the relative transformation to the tag is known. Each data collection, shown in \ref{fig:bimodal}, is ran through $1000$ trails and pose errors are measured.}
\label{fig:angle_result}
\end{figure*}

An alternative is to use all 4 detected corners as 4 pairs of point correspondences for the optimization. We projected the detected corners onto $D [ \boldsymbol{\hat{n}}, d]$ to get the coordinates $\boldsymbol{p} = [p_1, p_2, p_3, p_4]$ in the robot sensory frame. The corner coordinates $\boldsymbol{q} = [q_1, q_2, q_3, q_4]$ in the tag frame can be easily calculated since the tag is a square plane. We define the center of the tag as the origin, and the coordinates are simply the location of the corners on a Cartesian plane. Given these two sets of 3D point correspondences, the pose can be computed as a rigid body transformation estimation. Solving for the optimal transformation $[R, \boldsymbol{t}]$ requires minimizing a least squares error objective function given by:
\begin{IEEEeqnarray}{c}
[R, \boldsymbol{t}] = \argmin _{R \in SO(3), \boldsymbol{t}\in \mathbb{R}^3} \sum_{i=1}^{n} w_i \| R \boldsymbol{q_i} + \boldsymbol{t} - \boldsymbol{p_i}\| ^2
\IEEEeqnarraynumspace
\label{eq:rigid_body}
\end{IEEEeqnarray}
There are numerous approaches to solve Eq. \ref{eq:rigid_body} described in \citep{eggert1997estimating}. Since we have very few correspondences and they are assumed to be correct, it can be computed quickly using SVD:
\begin{IEEEeqnarray}{rCl}
& \bar{p} = \frac{1}{N} \sum_{i=1}^{N} p_i \qquad p_{ci} = p_i - \bar{p} \\
& \bar{q} = \frac{1}{N} \sum_{i=1}^{N} q_i \qquad q_{ci} = q_i - \bar{q} 
\end{IEEEeqnarray}
\begin{IEEEeqnarray}{rCl}
p_{c}^{\top}q_c &= U\Sigma V^\top \\
R & = VU^\top\\
\boldsymbol{t} & = \bar{q} - R\bar{p}
\end{IEEEeqnarray}
Here, $R$ and $t$ are the corresponding rotation and translation components of the the transformation. The above approach minimizes the least square error of the transformation and it is robust to small errors in the correspondences. The resulting pose obtained from the range data, although not accurate, provide a good approximation for the true pose. 

\subsection{Pose Refinement}

Lastly, the pose is refined by minimizing the reprojection error using the initial pose estimated from the previous step. The camera is assumed to be calibrated and the camera projection model $K$ is known. We denote $R^*$ and $\boldsymbol{t^{*}}$ to be the optimal pose in the constrained optimization function
\begin{IEEEeqnarray*}{rCl}
[R^*, \boldsymbol{t^{*}}] & = & \argmin _{R^*, \boldsymbol{t^{*}}} \sum_i^n \| (K [R^* | \boldsymbol{t^{*}}]) \boldsymbol{p_i} - \boldsymbol{y_i}\| ^2 \IEEEyesnumber \\
R^* & = & R (\Delta R) \IEEEyesnumber \\ 
\boldsymbol{t^{*}} & = & \boldsymbol{t} + R (\Delta \boldsymbol{t}) \IEEEyesnumber \\
\text{subject to:} \\ 
\Delta R & < & \Gamma _R \IEEEyesnumber \\
\Delta \boldsymbol{t} & < & \Gamma _t \IEEEyesnumber \\
\label{eq:refinement}
\end{IEEEeqnarray*}

Intuitively, the most optimal pose is the one with minimal reprojection error in the RGB space and align with the plane in the depth space. Therefore, the goal of the optimization is to find the local minimum closest to the initial estimation within allowable region $\Gamma$ as illustrated with Figure \ref{fig:optimization}. The key challenge is to determine the constrained region $\Gamma_R$ and $\Gamma_t$ such that it include a locally optimal pose and exclude the ambiguous pose. In most cases where the depth plane yields a good fit, this region should be small because the optimal pose is close to the initial estimate. When the depth sensor is noisy, the $\Gamma$ increases since the initial estimate might be far off. Thus, the constrained region $\Gamma$ is defined by the uncertainty in the initial estimate and it is characterized by the covariance of the plane parameters. In implementation, we used a trust-region optimization algorithm to bound the constraints. The scaling parameters for the covariance is empirically test to obtain the best results for our robot. 


\begin{figure}[h]
\centering
\includegraphics[width=\columnwidth, height=130px]{figs/viewing_angle_fig2}
\caption{Viewing Angle vs Error Percentage under different simulated noise level. The new RGBD based algorithm can resist noise in the RGB image and it vastly outperforms the original algorithm.}
\label{fig:viewing_result}
\end{figure}

The strength of this method is that it harness the benefits of RGB and depth information without explicitly assuming their relative accuracy. One advantage of RGBD sensors is that the camera and the depth sensor often work optimally with different constraints. In the example of Kinect, the RGB camera is sensitive to lighting and works poorly in scenes with low illumination. However, the time of flight depth sensor are unaffected by such a problem. On the hand, the time of flight sensor yield poor range results on surface edges, but the RGB camera works exceptionally well with edges where there is a high color contrast. 

% !TEX root = ../main.tex
\section{Experimental Results}
\label{sec:res}
The key problem we are trying to resolve is the localization accuracy of Apriltags in noisy situations. There are two major components we want to demostrate in this paper: first, we want to characterize the effect of perceptual ambiguity and noise on the Apriltag detection algorithms. Second, we want to test the resilience of our algorithm and show that it can obtain reasonable pose estimations under high level of noise. Finally, we briefly tested the runtime of the algorithm to show that it remains capable of real time detection. 

In our experiments, we measured the rotational and translational accuracy of the detections algorithms with respect to three different indepdent variables: viewing angles, distances, and lighting conditions. In all three experiments, we introduced 3 different levels of simulated detection noise into our images. We placed a standard camera calibration chessboard and an Apriltag of known size on a solid planar board. The apriltag has a fixed distance from the chessboard. This is used to compute the ground-truth pose for the tag. By using a large chessboard, we can detect the corners to a sub-pixel accuracy and compute accurate ground-truth poses unsusceptible to lighting and sensory noise.

\subsection{Viewing Angle}
The low localization accuracy caused by the perceptual ambiguity of the Apriltags is a non-linear function on the viewing angle of the tag. To characterize the effect, we placed the testing board on a table straight in front of the robot in a well lit room. Since the sensor is taller than the plane of the table, the robot has to slightly gazing down at it.  We rotated the testing board at a increment of 5 degrees from 0 degrees to 70 degrees. This is about the range in which the tag can be detected reliably given the camera resolution and distance. At each angle, we captured the RGB image, depth image, and detection outputs from the Apriltag algorithm. 

For each captured data collection, we introduced 3 levels of Gaussian noise of $\sigma = 0.2$, $\sigma = 0.5$, $\sigma = 1$  to the RGB image and computed the resulting tag pose. This is repeated for $1000$ trails at each noise level and the errors are computed for each trial. Figure [7] shows some of the results. 
\begin{figure}
\centering
\includegraphics[width=\columnwidth]{figs/viewing_angle_fig1}
\caption{Viewing Angle vs Error Percentage under different simulated noise level. The new RGBD based algorithm can resist noise in the RGB image and it vastly outperforms the original algorithm.
\scnote{This is not immediately important but these images should be PDF/EPS and not PNG. Also the figure organization could be better}}
\end{figure}

As the result shown in figure 7 indicates, the viewing angle has large effect on the rotation error. As we expected, the emperical results show a very clear bimodel distribution for the Apriltags at different viewing angles. The depth-sensor fused algorithm vastly outperforms the previous algorithm as it is not affected by the perceptual ambiguities. The small amount of the noise introduced to the data only cause a small rotational change around the true pose of the tag. In Figure[8] \scnote{I assume you know your figure numbering is messed up}, we thresholded all the poses based on their rotational errors and ploted the percentage of unacceptable poses at each viewing point. One interesting observation from the data is that, at most viewing angles, the magnitude of noise above a certain threshold has little effect on the locationlization accuracy. At most viewing angles, relatively small noises casuses a signficant accuracy decrease. 
\subsection{Distance}
In additional to the viewing angle, we caputred the images at different distances away from the camera. We moved the testing board perpendiualr to the sensor. 

The relationship between the distance and localization accuarcy is much more apparent. As the tag moves further away from the sensor, the number of pixels on the tag decreases. The perspective ambiguity becomes more apparent when there is only a small patch of pixels in the tag. 
\subsection{Lighting}
In the unsimulated environemnt, poor lighting is a large contributing factor of sensory noise. We tested the effect of the lights on our detection process by controling the background lighting. We captured the pictures under three different lighting conditions: dark, normal, and highly exposed. 

The Kinect sensor automatically adjusts the exposure settings to compensate for the low lighting. The pictures captured in the dark rooms still appears bright but much more grainy and apparent Gaussian noise in the image. Depth sensor and RGB sensor works optimally under different lighting conditions. In the dark setting, the depth sensor performs well. 


\begin{figure*}
\centering
\subfloat[RGB Image at 75$^{\circ}$]{\includegraphics[width=\columnwidth, height=160px]{figs/result_figs/rgb_frame0001} \label{RGB image}}
\subfloat[Distrubtion]{\includegraphics[width=\columnwidth, height=170px]{figs/result_figs/result_0001} \label{RGB image}}
\hfil
\subfloat[RGB Image at 40$^{\circ}$]{\includegraphics[width=\columnwidth, height=160px]{figs/result_figs/rgb_frame0005} \label{RGB image}}
\subfloat[Close]{\includegraphics[width=\columnwidth, height=170px]{figs/result_figs/result_0005} \label{RGB image}}
\hfil
\subfloat[RGB Image at 5$^{\circ}$]{\includegraphics[width=\columnwidth, height=160px]{figs/result_figs/rgb_frame0009} \label{RGB image}}
\subfloat[Close]{\includegraphics[width=\columnwidth, height=170px]{figs/result_figs/result_0009} \label{RGB image}}
\caption{The fiducial tag plane captured by the depth sensor}
\label{fig:calib}
\end{figure*}


% !TEX root = ../main.tex
\section{Conclusion}
\label{sec:conclusion}
% An important extension of the purposed method is to formally define the perceptual uncertainty produced by the pose estimation algorithm. For example, if the detector knows the pose estimated have large margins of error, it is useful to have some notion of uncertainty around the estimated pose. This is especially helpful inputs to any motion planing algorithms that takes uncertainty into consideration.

In this paper, we did a in depth analysis of the localization problem with Apriltags. We proposed a novel algorithm of using RGBD sensors to accurately compute the pose of Apriltags robust to noise. It is particularly suitable for robotic applications which requires precise poses such as manipulation, SLAM, and others. Furthermore, this technique can be easily generalized to other types of planar fiducial tags. Our implementation is fully open sourced and available at:

	http://somegithublink.com

{\footnotesize
\bibliographystyle{ieeetr}
\bibliography{bib/references}}

\end{document}

