% !TEX root = ../main.tex
\section{Introduction}
\label{sec:intro}
Detection and identification using artificial landmarks has long been used in augmented reality and computer vision applications. Over the last decade, there has been numerous marker systems for obtaining more accurate detections and information encodings. Specifcally, planar markers such as ARTag, ARToolkil, are among the most popular forms of fiducial markers.

Compared to markerless detection algorithms, these methods are simpler and produce much more reliable detections. There has been significant effort in improving the detection speed and encoding accuracy. They have yield great results for computer vision tasks that require high detection accuracies like camera calibrations, 3D reconstruction. Furthermore, they have gained popularity in the robotic community for having unique characteristics of having high detection rates and numerous encodings. For example, ARTags are commonly used to test SLAM systems or estimate ground truth for objects in manipulation and motion planning tasks. 

Despite all the imporvements in this field, obtaining accurate pose estimation from the tags remain a challenge. This is espcially important for generalizing the use of fiducial tags in robotic applications. While current detection algorithm yield great results under well conditioned or simulated environments, it is very difficult to obtain reliable poses in nosiey settings. For instances, when ARTags are used with low resolution camera or in poor lighting conditions, the system often produce poses with tremendous rotational errors as shown in Figure 2. In fact, we observe that the localization accuracy of the state of the art system perform significantly worse at difficult viewing angles or when there are noise in the scene. 

Based on the observation, we present two contributions in this paper. First, we conducted an in-depth analysis on the effect of various noises on pose estimation process. In particular, we characterize the persepctive ambiguity problem   In this paper, we present an algorithm that take advantage of the RGBD sensor to accurately estimate the pose from a single tag under noisy conditions. There are few key features to this algorithm: 
\begin{itemize}
\item The algorithm is generalizable to all square based fiducial tags.
\item The algorithm performs at worse as good as only using RGB images.
\end{itemize}